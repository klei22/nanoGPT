% titleofreport.tex
\documentclass{article}
\usepackage{titleofreport}

\usepackage[utf8]{inputenc}
\usepackage[T1]{fontenc}
\usepackage{hyperref}
\usepackage{url}
\usepackage{booktabs}
\usepackage{amsfonts}
\usepackage{nicefrac}
\usepackage{microtype}
\usepackage{xcolor}

\title{Rank-Aware Losses and Entropy-Aware Optimization for Improved Top-1 Accuracy}

\author{TODO: Author Names\\Affiliation\\\texttt{email@domain}}

\begin{document}

\maketitle

\begin{abstract}
We explore loss functions and optimizer modifications that explicitly encourage top-1 accuracy in transformer language models. Rank-distance and flatness-boost losses weight tokens by rank and prediction entropy, while an entropy-aware AdamW dynamically increases the learning rate on flat distributions. TODO: finalize quantitative gains and wording.
\end{abstract}

\section{Introduction}
Cross-entropy remains the de facto training objective for autoregressive language models, yet it does not directly optimize top-1 accuracy. Building on work such as focal loss~\cite{lin2017focal} and label smoothing~\cite{szegedy2016rethinking}, we investigate losses and optimizers that focus training on confidently predicting the correct next token. TODO: expand motivation and related work.

\section{Methods}
\subsection{Rank-distance loss}
This variant scales the per-token loss by how far the target logit ranks from the model's top prediction. TODO: include full formulation and scheduling strategy.

\subsection{Flatness-boost loss}
To discourage overly flat distributions, we amplify the loss when the maximum probability is low, akin to confidence penalties~\cite{pereyra2017regularizing}. TODO: add ablation details.

\subsection{Loss scheduling}
Our framework allows switching between loss variants during training. TODO: describe scheduling policies and implementation.

\subsection{Entropy-aware AdamW}
We propose an optimizer that boosts the learning rate when output entropy rises, drawing inspiration from entropy-aware methods~\cite{chaudhari2019entropy}. TODO: provide algorithmic pseudocode.

\section{Experiments}
We train 6-layer GPT models on the Minipile dataset using rotary embeddings and no absolute positional embeddings. Validation metrics include top-1 probability, correctness of the top prediction, and the rank of the target token. A preliminary summary appears in Table~\ref{tab:results}.

\begin{table}[ht]
    \centering
    \begin{tabular}{lccc}
        \toprule
        Loss Variant & Best Val Loss & Avg Top-1 Prob & Avg Top-1 Correctness \\
        \midrule
        focal & 1.57 & 0.54 & 0.44 \\
        rank\_distance & 1.62 & 0.53 & 0.44 \\
        flatness\_boost & 1.64 & 0.53 & 0.43 \\
        cross\_entropy & 1.65 & 0.52 & 0.42 \\
        \bottomrule
    \end{tabular}
    \caption{Preliminary validation metrics on the Minipile dataset. TODO: extend with additional losses, seeds, and variance estimates.}
    \label{tab:results}
\end{table}


TODO: detail training hyperparameters, compute costs, and additional baselines.

\section{Discussion}
Preliminary results suggest that rank-aware and entropy-aware techniques can modestly improve top-1 accuracy over vanilla cross-entropy. TODO: analyze failure cases and interactions between components.

\section{Conclusion}
We presented a suite of modifications targeting top-1 accuracy, spanning loss design and optimizer behavior. TODO: finalize conclusions with full experimental evidence.

\begin{ack}
TODO: acknowledgements and funding sources.
\end{ack}

\bibliographystyle{plainnat}
\bibliography{titleofreport}

\appendix
\section{Additional Results}
TODO: include expanded tables, training curves, and ablation studies.

\end{document}
