% titleofreport.tex
\documentclass{article}
\usepackage{titleofreport}

\usepackage[utf8]{inputenc}
\usepackage[T1]{fontenc}
\usepackage{hyperref}
\usepackage{url}
\usepackage{booktabs}
\usepackage{amsfonts}
\usepackage{nicefrac}
\usepackage{microtype}
\usepackage{xcolor}
\usepackage{listings}

\title{Rank-Aware Losses and Entropy-Aware Optimization for Improved Top-1 Accuracy}

\author{TODO: Author Names\\Affiliation\\\texttt{email@domain}}

\begin{document}

\maketitle

\begin{abstract}
We explore loss functions and optimizer modifications that explicitly encourage top-1 accuracy in transformer language models. Rank-distance and flatness-boost losses weight tokens by rank and prediction entropy, while an entropy-aware AdamW dynamically increases the learning rate on flat distributions. TODO: finalize quantitative gains and wording.
\end{abstract}

\section{Introduction}
Cross-entropy remains the de facto training objective for autoregressive language models, yet it does not directly optimize top-1 accuracy. Building on work such as focal loss~\cite{lin2017focal} and label smoothing~\cite{szegedy2016rethinking}, we investigate losses and optimizers that focus training on confidently predicting the correct next token. TODO: expand motivation and related work.

\section{Methods}
We implemented a suite of loss functions and an adaptive optimizer that emphasize confident top-1 predictions. The pseudocode below summarizes each variant's core update.

\subsection{Loss variants}

\paragraph{Cross-entropy (baseline)}
\begin{lstlisting}
function CROSS_ENTROPY(logits, target):
    return -log_softmax(logits)[target]
\end{lstlisting}
Standard maximum-likelihood objective.

\paragraph{Label smoothing}
\begin{lstlisting}
function LABEL_SMOOTHING(logits, target, s):
    p = one_hot(target)*(1-s) + s/|V|
    return -sum(p * log_softmax(logits))
\end{lstlisting}
Reduces overconfidence by distributing $s$ mass across other classes.

\paragraph{Focal loss}
\begin{lstlisting}
function FOCAL(logits, target, gamma):
    ce = CROSS_ENTROPY(logits, target)
    pt = exp(-ce)
    return (1-pt)^gamma * ce
\end{lstlisting}
Down-weights easy examples via $(1-p_t)^\gamma$.

\paragraph{Top-1 focus}
\begin{lstlisting}
function TOP1_FOCUS(logits, target, alpha):
    ce = CROSS_ENTROPY(logits, target)
    top = argmax(logits)
    penalty = 1 if top != target else 0
    return ce + alpha * penalty
\end{lstlisting}
Adds a margin if the model's prediction is not the target.

\paragraph{Top-1 margin}
\begin{lstlisting}
function TOP1_MARGIN(logits, target, m):
    ce = CROSS_ENTROPY(logits, target)
    max_other = max(logits[!=target])
    return ce + max(0, m - (logits[target] - max_other))
\end{lstlisting}
Encourages the target logit to exceed others by margin $m$.

\paragraph{Entropy penalty}
\begin{lstlisting}
function ENTROPY_PENALTY(logits, target, beta):
    ce = CROSS_ENTROPY(logits, target)
    H = -sum(softmax(logits)*log_softmax(logits))
    return ce + beta * H
\end{lstlisting}
Penalizes high-entropy predictions.

\paragraph{Top-1 ratio \textbf{(novel)}}
\begin{lstlisting}
function TOP1_RATIO(logits, target, beta):
    ce = CROSS_ENTROPY(logits, target)
    max_other = max(logits[!=target])
    ratio = exp(max_other - logits[target])
    return ce + beta * ratio
\end{lstlisting}
Pushes the target logit to dominate alternatives exponentially.

\paragraph{Rank-distance \textbf{(novel)}}
\begin{lstlisting}
function RANK_DISTANCE(logits, target, gamma):
    ce = CROSS_ENTROPY(logits, target)
    rank = 1 + count(logits > logits[target])
    scale = 1 + gamma * (rank-1)
    return ce * scale
\end{lstlisting}
Scales loss by how far the target ranks from top-1.

\paragraph{Flatness-boost \textbf{(novel)}}
\begin{lstlisting}
function FLATNESS_BOOST(logits, target, beta):
    ce = CROSS_ENTROPY(logits, target)
    H = entropy(softmax(logits)) / log(|V|)
    return ce * (1 + beta * H)
\end{lstlisting}
Amplifies loss on flat (high-entropy) distributions.

\subsection{Optimizer variant}

\paragraph{Entropy-aware AdamW \textbf{(novel)}}
\begin{lstlisting}
function ENTROPY_AWARE_ADAMW(g, m, v, H, lr, beta1, beta2, eps, w, wd, c):
    m = beta1*m + (1-beta1)*g
    v = beta2*v + (1-beta2)*g*g
    lr_eff = lr * (1 + c * H)
    w = w*(1-lr_eff*wd) - lr_eff * m / (sqrt(v)+eps)
\end{lstlisting}
Boosts the effective learning rate in proportion to batch entropy $H$.

\section{Experiments}
We train 6-layer GPT models on the Minipile dataset using rotary embeddings and no absolute positional embeddings. Validation metrics include top-1 probability, correctness of the top prediction, and the rank of the target token. A preliminary summary appears in Table~\ref{tab:results}.

\begin{table}[ht]
    \centering
    \begin{tabular}{lccc}
        \toprule
        Loss Variant & Best Val Loss & Avg Top-1 Prob & Avg Top-1 Correctness \\
        \midrule
        focal & 1.57 & 0.54 & 0.44 \\
        rank\_distance & 1.62 & 0.53 & 0.44 \\
        flatness\_boost & 1.64 & 0.53 & 0.43 \\
        cross\_entropy & 1.65 & 0.52 & 0.42 \\
        \bottomrule
    \end{tabular}
    \caption{Preliminary validation metrics on the Minipile dataset. TODO: extend with additional losses, seeds, and variance estimates.}
    \label{tab:results}
\end{table}

TODO: detail training hyperparameters, compute costs, and additional baselines.

\section{Discussion}
Table~\ref{tab:results} shows that losses emphasizing rank and entropy yield higher top-1 probability than vanilla cross-entropy. The novel rank-distance and flatness-boost losses each target different failure modes: the former penalizes low-ranked targets, while the latter reacts to diffuse predictions. Our entropy-aware AdamW further increases learning when predictions are flat. Combining these ideas---e.g., scheduling label smoothing early, then switching to rank-distance with entropy-aware AdamW---may tighten perplexity while boosting top-1 correctness. Future work can explore mixtures such as applying a top-1 ratio penalty alongside flatness-boost to jointly sharpen distributions and enforce logit dominance.

\section{Conclusion}
We presented a suite of modifications targeting top-1 accuracy, spanning loss design and optimizer behavior. TODO: finalize conclusions with full experimental evidence.

\begin{ack}
TODO: acknowledgements and funding sources.
\end{ack}

\bibliographystyle{plainnat}
\bibliography{titleofreport}

\appendix
\section{Additional Results}
TODO: include expanded tables, training curves, and ablation studies.

\end{document}
